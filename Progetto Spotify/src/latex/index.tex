\begin{DoxyAuthor}{Autori}
Claudio Ciccarone, Mattia De Donatis (in ordine alfabetico) 
\end{DoxyAuthor}
\begin{DoxyVersion}{Versione}
1.\+0.\+4 
\end{DoxyVersion}
\begin{DoxyDate}{Data}
19/05/2018 
\end{DoxyDate}
\begin{DoxyCopyright}{Copyright}
G\+NU Pubblic license
\end{DoxyCopyright}
\hypertarget{index_intro_sec}{}\section{Introduzione}\label{index_intro_sec}
Questo codice è stato scritto per un progetto scolastico in C.~\newline
 Il programma imita la gestione utenti e artisti della famosa piattaforma di streaming audio Spotify.~\newline
 -\/\+Il programma è stato implementato inserendo un menù principale dal quale si possono accedere a due sottomenù\+:
\begin{DoxyEnumerate}
\item Gestione artisti\+: Permette di gestire gli artisti. Più nello specifico permette di\+:
\begin{DoxyEnumerate}
\item Visualizzare tutte le informazioni sugli artisti
\item Aggiungere un nuovo artista
\item Modificare il profilo dell\textquotesingle{}artista
\item Eliminare il profilo dell\textquotesingle{}artista
\end{DoxyEnumerate}
\item Gestire gli utenti\+: Permette di interfacciare l\textquotesingle{}utente al programma; da qui l\textquotesingle{}utente può\+:
\begin{DoxyEnumerate}
\item Registrarsi e quindi creare un nuovo profilo utente
\item Accedere al programma inserendo Nickname e password; effettuato l\textquotesingle{}accesso, l\textquotesingle{}utente potrà\+:
\begin{DoxyEnumerate}
\item Visualizzare il suo profilo
\item Modificare le sue preferenze ascoltando gli artisti o inserendo \char`\"{}\+Mi piace\char`\"{} o \char`\"{}\+Non mi piace\char`\"{}
\item Modificare il suo profilo
\item Disiscriversi
\item Effettuare il Log-\/\+Out
\end{DoxyEnumerate}
\end{DoxyEnumerate}
\item Chiudere Spotify~\newline
 Per ulteriori informazioni riguardo le funzioni del programma leggere la documentazione.
\end{DoxyEnumerate}\hypertarget{index_comp_sec}{}\section{Compilazione}\label{index_comp_sec}
Il codice per essere compilato a bisogno di un qualsiasi compilatore C e necessita di includere la libreria \hyperlink{funzioni_8h}{funzioni.\+h} in questo modo\+: include \char`\"{}funzioni.\+h\char`\"{} \begin{DoxyWarning}{Avvertimento}
Eseguire il programma solo su Windows e inserire l\textquotesingle{}eseguibile in una cartella chiamata \char`\"{}\+Progetto Spotify\char`\"{}.~\newline
 I file si devono trovare in una sottocartella di \char`\"{}\+Progetto Spotify\char`\"{} chiamata \char`\"{}\+File\char`\"{}. 
\end{DoxyWarning}
\hypertarget{index_file_sec}{}\section{File}\label{index_file_sec}
I file usati nel progetto sono\+: \hypertarget{index_file1}{}\subsection{Source principale}\label{index_file1}
Formato da \hyperlink{main_8c}{main.\+c} \hypertarget{index_file2}{}\subsection{Source funzioni}\label{index_file2}
Formato da \hyperlink{funzioni_8c}{funzioni.\+c} -\/ \hyperlink{funzioni__artista_8c}{funzioni\+\_\+artista.\+c} -\/ \hyperlink{funzioni__utente_8c}{funzioni\+\_\+utente.\+c} \hypertarget{index_file3}{}\subsection{File Header}\label{index_file3}
Formato da \hyperlink{funzioni_8h}{funzioni.\+h} \hypertarget{index_file4-5-6}{}\subsection{File dati}\label{index_file4-5-6}
\hypertarget{index_artisti}{}\subsubsection{Artisti}\label{index_artisti}
Il file artisti.\+txt contiene tutti gli artisti memorizzati nel Database.~\newline
 -\/I dati sono memorizzati nel seguente ordine (separati da una virgola)\+:
\begin{DoxyEnumerate}
\item Codice artista\+: Es.\+0001 Codice formato da 4 cifre che identifica univocamente un artista. Non può esistere un codice 0000.
\item Nome artista\+: Es. Depeche Mode Indica il nome dell\textquotesingle{}artista o del gruppo.
\item Nome casa discografica\+: Es. Virgin Radio Indica il nome della casa discografica dell\textquotesingle{}artista.
\item Nazionalità\+: Es. Inglese Indica la nazionalità dell\textquotesingle{}artista o del gruppo.
\item Anno\+: Es.\+1980 Indica l\textquotesingle{}anno di inizio attività dell\textquotesingle{}artista o del gruppo.
\item Numero Ascolti\+: Indica il numero degli ascolti complessivi dell\textquotesingle{}artista.
\item Numero \char`\"{}\+Mi piace\char`\"{}\+: Indica il numero di \char`\"{}\+Mi piace\char`\"{} complessivi ricevuti dagli utenti, dell\textquotesingle{}artista.
\item Genereri\+: Es. Rock Indica i/il genere/i dell\textquotesingle{}artista. Questi possono essere anche più di uno e saranno separati da una virgola. 
\end{DoxyEnumerate}\hypertarget{index_utenti}{}\subsubsection{Utenti}\label{index_utenti}
Il file utenti.\+txt contiene tutti gli utenti memorizzati nel database.~\newline
 -\/I dati sono memorizzati nel seguente ordine (separati da una virgola)\+:
\begin{DoxyEnumerate}
\item Nickname\+: Es. Cioscos Nome utilizzato dall\textquotesingle{}utente per accedere al programma.
\item Password\+: Utilizzata per accedere al programma dall\textquotesingle{}utente. \begin{DoxyWarning}{Avvertimento}
La password non potrà mai essere visualizzata dal programma ma potrà esserlo, invece, dal file utenti.\+txt
\end{DoxyWarning}

\item Nome\+: Nome dell\textquotesingle{}utente
\item Cognome\+: Cognome dell\textquotesingle{}utente
\item Data di nascita\+: Data di nascita memorizzata nell\textquotesingle{}ordine G\+G/\+M\+M/\+A\+A\+AA
\item Data di iscrizione\+: Data di iscrizione memorizzata nell\textquotesingle{}ordine G\+G/\+M\+M/\+A\+A\+AA. 
\end{DoxyEnumerate}\hypertarget{index_preferenze}{}\subsubsection{Preferenze}\label{index_preferenze}
Il file preferenze.\+txt contiene tutte le preferenze degli utenti.~\newline
 -\/I dati sono memorizzati nel seguente ordine (separati da uno slash)\+:
\begin{DoxyEnumerate}
\item Nickname\+: Nickname utente che ha inserito la preferenza
\item Codice Artista\+: Codice dell\textquotesingle{}artista al quale è stata inserita la preferenza.~\newline
 Dopo questo codice è presente una virgola seguita da un numero che può indicare\+:~\newline

\begin{DoxyEnumerate}
\item 1\+: Indica che l\textquotesingle{}artista è stato solamente ascoltato dall\textquotesingle{}utente.
\item 2\+: Indica che l\textquotesingle{}artista è stato ascoltato dall\textquotesingle{}utente e che quest\textquotesingle{}ultimo ha gradito l\textquotesingle{}ascolto (incremento del numero di \char`\"{}\+Mi piace\char`\"{})
\item 3\+: Indica che l\textquotesingle{}artista è stato ascoltato dall\textquotesingle{}utente e che quest\textquotesingle{}ultimo non ha gradito l\textquotesingle{}ascolto. 
\end{DoxyEnumerate}
\end{DoxyEnumerate}